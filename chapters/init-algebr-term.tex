\chapter{Initial algebras and terminal coalgebras}
\label{chap:init-algebr-term}

\section{Natural numbers}
\label{sec:natural-numbers}

Following Peano we define the natural numbers as inductively generated from $0:\Nat$ and $\suc : \Nat \to \Nat$. We can define functions like addition
$\_+\_ : \Nat \to \Nat \to \Nat$ by recursion:
\begin{align*}
  0 + m & = m\\
  \suc\,n + m & = \suc\,(n + m)
\end{align*}
and we can prove properties such that for all $n : \Nat$ we have that $n + 0 = n$ by induction:
\begin{description}
\item[$n=0$] $0 + 0 = 0$ follows from the definition of $\_+\_$.
\item[$n = \suc\,m'$] we assume that the statement holds for $m$:
  \begin{align*}
    n + 0 
    & = \suc\,m + 0 \\
    & = \suc\,(m + 0) & \mbox{Defn of $\_+\_$}\\
    & = \suc\,m & \mbox{Ind.hypothesis} \\
    & = n
  \end{align*}
\end{description}

To categorify natural numbers we observe that we have two morphisms
\begin{align*}
  0  & : 1 \to \Nat \\
  \suc & : \Nat \to \Nat
\end{align*}
and given a set $A$ and $z:A$ or equivalently $z : 1 \to A$ and $s : A \to A$ we can define $\It_A\,z\,s : \Nat \to A$ as:
\begin{align*}
\It_A\,z\,s\,0 & = z\\
\It_A\,z\,s\,(\suc\,n) & = s\,(\It_A\,z\,s\,n)
\end{align*}
this is the unique morphism that makes the following diagram commute
\[\begin{tikzcd}[row sep = large]
& A \arrow[r,"s"] & A \\
1 \arrow[ur,"z"] \arrow[r,"0"] & \Nat \arrow[r,"\suc"] \arrow[u, dashrightarrow,"\It_A\,z\,s" right] & \Nat  \arrow[u, dashrightarrow,"\It_A\,z\,s" right]
\end{tikzcd}\]  
The uniqueness of $\It_A\,z\,s$ can be shown by induction. We assume that there is another function $h : \Nat \to A$ that makes the digram commute, i.e. 
\begin{align*}
h \circ 0 & = z\\
h \circ \suc & = s \circ h
\end{align*}
We now show that $h = \It_A\,z\,s$. Using extensionality it is enough to show $h\,n = \It_A\,z\,s\,n$ for all $n:\Nat$. We show this by induction:
\begin{description}
\item[$n=0$] 
  \begin{align*}
    h\,0 & = h \circ 0 \\
    & = z\\
    & = \It_A\,z\,s\,0
  \end{align*}
\item[$n = \suc\,n'$] 
  \begin{align*}
    h\,(\suc\,n') & = (h \circ \suc) \, n' \\
    & = (s \circ h)\,n' \\
    & = s (h \, n')\\
    & = s (\It_A\,z\,s\,n') & \mbox{by ind.hyp.} \\
    & = \It_A\,z\,s\,(\suc\,n')
  \end{align*}
\end{description}

This leads directly to the definition of a \emph{Natural Number Object (NNO)} in a category $\cat{C}$ with a terminal object: it is given by an object $\Nat : \obj{\cat{C}}$ and two morphisms $0 : \homC{C}{1}{\Nat}$ and $\suc : \homC{\cat{C}}{\Nat}{\Nat}$ such for any object $A$ and $z : \homC{C}{1}{A}$ and $s:\homC{C}{A}{A}$ there is a unique morphism $\It_A\,z\,s$ that makes the above diagram commute.

We can derive the addition function exploiting also cartesian closure. We choose $A = \Nat \to \Nat$ and
\begin{align*}
& z : 1 \to (\Nat \to \Nat) \\
& z\,()\,n = n \\
& s : (\Nat \to \Nat) \to (\Nat \to \Nat) \\
& s\,f\,n = \suc\,(f\,n)
\end{align*}
We can now define $\_+\_ = \It_{\Nat\to\Nat}\,z\,s : \Nat \to (\Nat \to \Nat)$.
\begin{Exercise}
  Convince yourself that this computes the same function as the previous definition of $\_+\_$ using pattern maching.
\end{Exercise}

We can prove that $n + 0 = 0$ using the unicity of $\It$ instead of induction. We observe that both $\id_\Nat$ and $\_+0 : \Nat \to \Nat$ make the following diagram commute:
\[\begin{tikzcd}[row sep = large]
& \Nat \arrow[r,"\suc"] & \Nat \\
1 \arrow[ur,"0"] \arrow[r,"0"] & \Nat \arrow[r,"\suc"] \arrow[u, shift left,"\id_\Nat" left] 
\arrow[u,shift right, "\_+0" right]
& \Nat  \arrow[u,shift left,"\id_\Nat" left]\arrow[u,shift right,"\_+0" right]
\end{tikzcd}\]  
And hence by uniqueness $\id_\Nat = \_+0$, by applying both sides to $n:\Nat$ we obtain 
$n = \id\,n = (\_ + 0)\,n = n + 0$. 
\begin{Exercise}
  Verify the claim that both functions make the diagram commute.
\end{Exercise}

One basic function on natural numbers is the \emph{predecessor}, i.e. the inverse to the successor. Initially we may think that we should define a function $\pred : \Nat \to \Nat$ with $\pred\,(\suc\,n) = n$. But what should $\pred\,0$ be? We could arbitrarily set $\pred\,0 = 0$ but this seems a very unprincipled move. It seems better to say that $\pred$ should return an \emph{error} when applies to $0$. That is as well known in functional programming, we should use the \emph{Maybe}-monad
\footnote{Even though we don't actually know yet what a monad is}
and use $\Maybe\,A = 1 + A$ as the result type. That is we define $\pred : \Nat \to 1 + \Nat$ as
\begin{align*}
\pred\,0 & = \inj_1\,() \\
\pred\,(\suc\,n) & = \inj_2\,n
\end{align*}
However, we cannot translate this definition directly into an application of $\It$ since in the successor case we referred directly to $n$ while $\It$ just gives us the recursive result. However, the following alternative definition works:
\begin{align*}
\pred\,0 & = \inj_1\,() \\
\pred\,(\suc\,n) & = \case{\inj_2\,0}{\inj_2 \circ \suc}
\end{align*}
The idea is that the 2nd definition computes the predeccessor by delaying the constructors by one step.
\begin{Exercise}
Verify in detail that both definitions of $\pred$ compute the same function.                     
\end{Exercise}
The second definition can be translated into 
\begin{align*}
& \pred : \Nat \to 1+\Nat \\
& \pred = \It_{1+\Nat}\,(\inj_1\,())\, \case{\inj_2\,0}{\inj_2 \circ \suc}
\end{align*}

$\pred$ is the inverse of the constructors $0$ and $\suc$. We can make this precise by packaging $0$ and $\suc$ into one function 
\begin{align*}
& \inn : 1+\Nat \to \Nat \\
& \inn = \case{0}{\suc} 
\end{align*}
This is an instance of Lambek's lemma which we will verify in the general case later. 
\begin{Exercise}
  Prove by induction or by using initiality that $\pred$ and $\inn$ are an isomorphism, i.e.
  \begin{align*}
    \pred \circ \inn & = \id_{1+\Nat} \\
    \inn \circ \pred & = \id_\Nat
  \end{align*}
\end{Exercise}

\section{Initial algebras}
\label{sec:initial-algebras}

Natural numbers are an instance of the general concept of an initial algebra which are the categorical counterpart of what we call an inductive definition, that is a datatype which is generated by constructors. An initial algebra is specified by an endofunctor $T$. In the case of natural numbers this was the functor $T_\Nat : \Set \to \Set$ defined as $T_\Nat\,X = 1+X$. 

We give a general definition: Given an endofunctor $T : \cat{C} \to \cat{C}$ an initial $T$-algebra is given by an object $\mu T : \obj{\cat{C}}$ and a morphism $\inn_T : : \homC{C}{T\,\mu\,T}{\mu\,T}$ such that for any $T$-algebra, that is a pair of $A : \obj{\cat{C}}$ and a morphism 
$f : \homC{C}{T\,A}{A}$ there is a unique morphism $\It_T \,f: \cat{C}{\mu T}{A}$
\footnote{I leave the carrier $A$ implicit since it can be inferred from $f$.}
that makes the following diagram commute:
\[\begin{tikzcd}
T\,A \arrow[r,"f"]  & A \\
T\,(\mu T) \arrow[u, "T\, (\It_T \, f"]  \arrow[r,"\inn_T"] & \mu T \arrow[u,"\It_T \, f" right,dashrightarrow]
\end{tikzcd}\]  
Indeed, the initial algebra is just the initial object in the category of $T$-algebras, whose objects are $T$-algebras and given $T$-algebras $f : \homC{C}{T\,A}{A}$ and $g : \homC{C}{T\,B}{B}$ a morphism is given by a morphism $h : \homC{C}{A}{B}$ such that the following diagram commutes:
\[\begin{tikzcd}
T\,B \arrow[r,"g"] & B \\
 T\,A \arrow[r,"f"] \arrow[u,"T\,h"]  & A \arrow[u,"h"] 
\end{tikzcd}\]  
Identity and composition is given by identity and composition in $\cat{C}$, it is easy to see that the corresponding diagrams commute. 

\begin{Exercise}
Show that $\Nat$ is the initial $T_\Nat$ algebra with $T_\Nat : \Set \to \Set$ defined as $T_\Nat\,X = 1+X$. 
\end{Exercise}

We can understand the type of lists in a similar fashion. $\List$ are given by two constructors (pronounced nil and cons):
\begin{align*}
[] & : \List\,A \\
\_\cons\_ & : A \to \List\,A \to \List\,A
\end{align*}
We can transform this into one constructor function by first uncurrying
\begin{align*}
1 & \to \List\,A \\
A \times \List\,A & \to \List\,A
\end{align*}
and then merging them into one function using coproducts:
\begin{align*}
1 + A \times \List\,A & \to \List\,A
\end{align*}
Hence $\List\,A$ is the initial algebra of the functor $T_{\List\,A}:\cat{\Set} \to \cat{\Set}$ with $T_{\List\,A}\,X = 1 + A \times X$.
\begin{Exercise}
Define the function $\rev_A : \List\,A \to \List\,A$ using only the iterator. \emph{Hint:} it is useful to define an auxilliary function 
$\mathrm{snoc} : A \times\List\,A \to A$ that appends a single element at the end of a list using the iterator. 

Show that this function is idempotent, that is $\rev_A \circ \rev_A = \id_{\List\,A}$ using only initiality. 
\end{Exercise}

\begin{Exercise}
What are the functors defining 
\begin{enumerate}
\item unlabelled binary trees,
\item binary trees whose leaves are labelled with $A:\Set$,
\item binary trees whose nodes  are labelled with $A:\Set$,
\end{enumerate}
\end{Exercise}

Since we already know that $\List$ is a functor, what is the initial algebra of the list functor? This has a constructor
\begin{align*}
\inn_\List : \List \,(\mu\,\List) \to \mu\,\List
\end{align*}
Indeed, this is the type of finitely branching trees, that is any node has a list of subtrees. We don't need to include leaves explicitely because we can have a node with an empty list of subtrees. This type is also called \emph{rose trees}.

We can also construct infinitely branching trees which are the initial algebra of the functor $T\,X = 1 + \Nat \to X$. Here a node is given by a function that assigns to every natural number a subtree. This raises the question wether every functor on sets should have an initial algebra. We note that obviously paradoxical cases like $T\,X = X \to X$ are ruled out because they don't give rise to a functor. However, depending on our foundations even positive hence functorial definitions like $T\,X = (X \to \Bool) \to \Bool$ are problematic. If we work in a classical setting then this is obviously paradoxical because it gives us a fixpoint of the double powerset functor and from this we can derive a contradiction using diagonalisation.
\begin{Exercise}
Show that having an initial algebra of $T : \Set \to \Set$ with $T\,X = (X \to \Bool) \to \Bool$ leads to a contradiction if we assume $\Prop = \Bool$ (classical logic). Assume that $\inn_T : T (\mu\,T) \to \mu\,T$ is an isomorphism
\footnote{We are going to prove this in the next section.}.
\emph{Hint:} Derive a retraction, i.e. a function $\phi : (\mu\,T \to \Bool) \to \mu\,T$ that has a left inverse. Show that such a retract cannot exists using Cantor's diagonalisation.
\end{Exercise}

\section{Lambek's lemma}
\label{sec:lambeks-lemma}

We return to the predecessor function and show that the constructor morphism $\inn_T : \homC{C}{T\,(\mu\,T)}{\mu T}$ is an isomorphism for any initial $T$-algebra. The inverse is given by 
\begin{align*}
\out_T : \homC{C}{\mu T}{T\,(\mu\,T)} \\
\out_T = \It_T\,(T\,\inn_T)
\end{align*}
which generalizes our definition of $\pred$. To see that this is an isomorphism, we use the following diagram:
\[\begin{tikzcd}
T\,(\mu T) \arrow[r,"\inn_T"] & \mu T\\
T\,(T\,(\mu\,T)) \arrow[r,"T\,\inn_T"]  \arrow[u,"T\,\inn_T" right]& T\,(\mu\,T) \arrow[u,"\inn_T" left]\\
T\,(\mu T) \arrow[u, "T\, \out_T" right]  \arrow[r,"\inn_T"] \arrow[uu,"T\,\id = \id",bend left = 60] & \mu T \arrow[u,"\out_T" left] \arrow[uu,"\id" right, bend right = 60]
\end{tikzcd}\]  
The lower square commutes due to the definition of $\out_T$, the upper square commutes trivially. On the other hand the whole square also commutes with $\id$ which on the left equals $T\,\id$. However, since there is at most one algebra morphism, we know that $\inn_T \circ \out_T = \id$. Now we can reason that :
\begin{align*}
  \out_T \circ \inn_T & = T\, \inn_T \circ T\, \out_T & \mbox{lower square} \\
  & = T\,(\inn_T \circ \out_T) \\
  & = T \, \id \\
  & = \id
\end{align*}
Hence we have verified that $\inn_T$ and $\out_T$ are an isomorphism.

\section{Streams}
\label{sec:terminal-coalgebras}

The dual of initial algebras, terminal coalgebras, turn out to be useful as well. An example of a coinductive type is the type of streams $\Stream\,A : \Set$ this are infinite sequences of elements of $A:\Set$, that is 
\begin{align*}
  [a_0,a_1,a_2 \dots  : \Stream\,A
\end{align*}
with $a_i : A$.
Streams can be understood via destructors, that is for for any stream we can compute its head and its tail:
\begin{align*}
\head & : \Stream\,A \to A \\
\tail & : \Stream\,A \to \Stream\,A
\end{align*}
To construct streams we need a coiterator, that is given $X : \Set$ and functions 
\begin{align*}
  h & : X \to A \\
  t & : X \to X
\end{align*}
we obtain a function:
\begin{align*}
\CoIt\,h\,t & : X \to \Stream\,A
\end{align*}
which is given by the copatterns:
\begin{align*}
\head\,(\CoIt\,h\,t\,x) & = h\,x \\
\tail\,(\CoIt\,h\,t\,x) & = \CoIt\,h\,t\,(t\,x)
\end{align*}
$\CoIt\,h\,t $ is the unique function that makes the following diagram commute:
\[\begin{tikzcd}
    &  X  \arrow[dl,"h" left]\arrow[r,"t"]\arrow[d, dashrightarrow,"\CoIt\,h\,t"] & X \arrow[d, dashrightarrow,"\CoIt\,h\,t"]\\
A & \arrow[l,"\head"]\Stream\,A \arrow[r,"\tail"] & \Stream|,A
\end{tikzcd}\]  
As an example we can define the function $\from : \Nat \to \Stream\,\Nat$ which generates a stream of natural numbers starting with the input. I.e.
\begin{align*}
\from\,n & = [ n , n+1 , n+2 , \dots 
\end{align*}
We can define $\from$ via copatterns
\begin{align*}
\head\,(\from\,n) & = n \\
\tail\,(\from\,n) & = \from\,(\suc\,n)
\end{align*}
and this can be turned into an application of the coiterator:
\begin{align*}
\from & = \CoIt\,\id_\Nat\,\suc
\end{align*}
$\Stream$ is a functor, that is given a function $f:A \to B$ we can define
\begin{align*}
\Stream\,f & : \Stream\,A \to \Stream\,B \\
\head\,(\Stream\,f\,\vec{a}) & = f\,(\head\,a) \\
\tail\,(\Stream\,f\,\vec{a}) & = \Stream\,f\,(\tail\,\vec{a})
\end{align*}
\begin{Exercise}
  Define $\Stream\,f$ using only the coiterator.
\end{Exercise}
% The dual of induction is coinduction. This is usually formulated using the concept of a bisimulation: A relation 
% $\_R\_ \subseteq \Stream\,A \times \Stream\,A$ is called a bisimulation if given $\vec{a} R \vec{b}$ then 
% $\head\,\vec{a} = \head\,\vec{b}$ and $\tail\,\vec{a} R \tail\,\vec{b}$. Coinduction states that bisimilar streams are equal, i.e. if $\vec{a} R \vec{b}$ then $\vec{a} = \vec{b}$.

% As an example consider the equation
% \begin{align*}
% \from\,(\suc\,n) = \Stream\,\suc\,(\from\,n)
% \end{align*}
% We define $\vec{m} R \vec{n}$ by saying that there exists a number $k$ such that $\vec{m} = \from\,(\suc k)$ and $\vec{n} = \Stream\,suc\,
We can use uniqeness to prove equations, for example we want to verify
\begin{align*}
\from\,(\suc\,n) = \Stream\,\suc\,(\from\,n)
\end{align*}
for any number $n$. We turn this into an equation for functions $\Nat \to \Stream\,\Nat$:
\begin{align*}
  \from \circ \suc = \Stream\,\suc \circ \from
\end{align*}
We observe that 
\begin{align*}
  \head\,(\from\,(suc\,n)) & = \suc n\\
  \head\,(\Stream\,\suc\,(\from\,n)) & = \suc\,(\head\,(\from\,n) \\
  & = \suc\,n\\
  \tail\, (\from\,(suc\,n)) & = \from\,(\suc\,(\suc\,n))\\
  \tail\, (\Stream\,\suc\,(\from\,n)) & = \Stream\,\suc\,(\tail\,(\from\,n))\\
                                        & = \Stream\,\suc\,(\from\,(\suc\,n))
\end{align*}
That is both side are equal to $\CoIt\,\suc\,\suc$ and hence equal due to uniqueness. 

\begin{Exercise}
  Prove that $\Stream\,A$ is isomorphic to $\Nat \to A$.
\end{Exercise}

\section{Terminal coalgebras}
\label{sec:terminal-coalgebras}

In general a terminal coalgebra of an endofunctor $T : \cat{C} \to \cat{C}$ is given by an object $\nu T : \obj{\cat{C}}$ and a morphism $\out_T : : \homC{C}{\nu\,T}{T\,(\nu\,T)}$ such that for any $T$-coalgebra, that is a pair of $A : \obj{\cat{C}}$ and a morphism 
$f : \homC{C}{A}{T\,A}$ there is a unique morphism $\CoIt_T \,f: \cat{C}{A}{\nu T}$
that makes the following diagram commute:
\[\begin{tikzcd}
A \arrow[d,"\CoIt_T \, f" right,dashrightarrow] \arrow[r,"f"]  & T\,A \arrow[d, "T\, (\CoIt_T \, f"]\\
\nu\,T  \arrow[r,"\out_T"] & T\,(\nu T) 
\end{tikzcd}\]  
Indeed, the terminal coalgebra is just the terminal object in the category of $T$-coalgebras, whose objects are $T$-coalgebras and given $T$-coalgebras $f : \homC{C}{A}{T\,A}$ and $g : \homC{C}{B}{T\,B}$ a morphism is given by a morphism $h : \homC{C}{A}{B}$ such that the following diagram commutes:
\[\begin{tikzcd}
 A \arrow[d,"h"] \arrow[r,"f"]  & T\,A \arrow[d,"T\,h"]\\
 B \arrow[r,"g"] & T\,B 
\end{tikzcd}\]  
As for algebras, Identity and composition is given by identity and composition in $\cat{C}$, it is easy to see that the corresponding diagrams commute. 

It is clear that $\Stream\,A$ is the terminal coalgebra of the functor $T_{\Stream\,A}\,X = A \times X$. We can look at initial algebras and terminal coalgebras for the same functor. In the case of $T_{\Stream\,A}$  this is not very interesting because the initial algebra is just the empty set. However, natural numbers $T_\Nat\,X = 1+X$ and lists $T_{\List\,A}\,X = 1 + A \times X$ are more interesting: here the terminal coalgebras are called the conatural numbers and colists: they include both finite elements and infinite elements. 

\begin{Exercise}
  Show that the initial algebra of $T_{\Stream\,A}\,X = A \times X$ is isomorphic to the empty set
\end{Exercise}

\begin{Exercise}
  Define addition for conatural numbers ($\Nat^\infty = \nu X.1+X$) using only the coiterator.
\end{Exercise}

\begin{Exercise}
  Show explicitely that Lambek's lemma hold for terminal coalgebras.
\end{Exercise}


% \begin{Exercise}
%   Show that there is always a monomorphism from the initial algebra to the terminal coalgebra.
% \end{Exercise}