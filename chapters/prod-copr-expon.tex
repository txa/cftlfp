\section{Products, coproducts and exponentials}
\label{sec:prod-copr-expon}

\subsection{Products}
\label{sec:products}

The product of two sets $A,B : \Set$ is $A\times B$ its elements are tuples, that is given $a:A$ and $b:B$ we can form $(a,b) : A \times B$ which are the only elements of this type. We can define the projections:
\begin{align*}
\pi_1 : A \times B \to A \\
\pi_1\,(a,b) = a \\
\pi_2 : A \times B \to A \\
\pi_2\,(a,b) = b
\end{align*}
Moreover given another set $C:\Set$ and functions $f : A \to C$ and $g : B \to C$ we define
\begin{align*}
\pair{f}{g} : C \to A \times B \\
\pair{f}{g}\,c = (f\,c,g\,c)
\end{align*}
this is the unique function that makes the following diagram commute 
\[\begin{tikzcd}[row sep = large]
& C \arrow[dl,"f"] \arrow[d, dashrightarrow,"\pair{f}{g}"] \arrow[dr,"g"]\\
A & A \times B \arrow[r,"\pi_2"] \arrow[l,"\pi_1"]& B
\end{tikzcd}\]
The uniqueness is indicated by the dashed arrow. 

By unique we mean that for any other function $h : C \to A \times B$ which makes this diagram commute we have that $h = \pair{f}{g}$. Using extensionality it is sufficient to show that $h\,x = \pair{f}{g}\,x$ :
\begin{align*}
\pair{f}{g}\,x
& = (f\,x,g\,x) \\
& = ((\pi_1 \circ h)\,x, \pi_2 \circ h)\,x)\\
& = (\pi_1 \,( h\,x), \pi_2\,(h\,x))\\
& = (\lambda z.(\pi_1\,z,\pi_2\,z))\circ h\\
\end{align*}
Using the fact that all elements of $A \times B$ are tuples it is easy to see that $\lambda z.(\pi_1\,z,\pi_2\,z)) = \lambda z.z = \id$ and hence
\begin{align*}
(\lambda z.(\pi_1\,z,\pi_2\,z))\circ h
& = \id \circ h \\
& = h
\end{align*}
Moreover, $\pair{\_}{\_}$ is natural, this means that $\pair{f}{g} \circ h = \pair{f \circ h}{g \circ h}$:
\begin{align*}
(\pair{f}{g} \circ h)\,x
& = \pair{f}{g}\,(h\,x) \\
& = (f\,(h\,x),g\,(h\,x)) \\
& = ((f\circ h)\,x),(g\circ h)\,x)) \\
& = \pair{f \circ h}{g \circ h}\,x
\end{align*}

We abstract from products in $\cat{Set}$ and say that in any category $\cat{C}$ a product $A \times B$ of two objects $A,B$ is given by two morphisms $\pi_1 : \homC{C}{A\times B}{A}$ and $\pi_2 : \homC{C}{A\times B}{B}$ and for any pair of morphisms $f:\homC{C}{C}{A}$ and 
$g : \homC{C}{C}{B}$ a unique morphism $\pair{f}{g} : \homC{C}{C}{A \times B}$ that makes the following diagram commute:
\[\begin{tikzcd}[row sep = large]
& C \arrow[dl,"f"] \arrow[d, dashrightarrow,"\pair{f}{g}"] \arrow[dr,"g"]\\
A & A \times B \arrow[r,"\pi_2"] \arrow[l,"\pi_1"]& B
\end{tikzcd}\]
Moreover this assignment is natural, i.e. $\pair{f}{g} \circ h = \pair{f\circ h}{g\circ h}$.

\begin{question}
  Does the category $\cat{\omega}$, which corresponds to $\leq$ on natural numbers, have products? In general what are products for a preorder?
\end{question}

As for terminal and initial objects we can show that products are unique up to isomorphism. This also implies that  $A\times B \cong B \times A$ because it is easy to see that $B\times A$ is a product of $A$ and $B$.
% do this explicitly?

We have only defined binary products, what about ternary products? There are at least two ways to derive them from binary products $A \times (B\times C)$ and $(A\times B)\times C$. 
\begin{Exercise}
  Show that $A \times (B\times C) \cong (A\times B)\times C$. 
\end{Exercise}
\begin{Answer}
  We'll call these morphisms $\phi\colon A \times (B\times C) \to (A\times B)\times C$ and $\phi\inv$ the other way around.

  To define $\phi$, notice that its codomain is the product of two objects: $(A\times B)$ and $C$. To define a morphism \emph{into} a product, we need to give a pair of morphisms:
  \[ \phi=[f,g] \text{ for } f\colon A\times(B\times C)\to (A\times B)\text{ and }g\colon A\times(B\times C)\to C\]
  But $f$ itself goes into a product, so to define $f$, we also need a pair of functions $f=[p,q]$, where $p\colon A\times(B\times C)\to A$ and $q\colon A\times(B\times C)\to B$. Now we can define these explicitly:

  \begin{tabular}{@{} lll @{}}
    $p := \pi_1$
        &\hspace{2cm}\;&   \begin{tikzcd}
          A\times(B\times C) \arrow{r}{\pi_1} & A
            \end{tikzcd}    \\
    $q := \pi_1\circ\pi_2$
        &&   \begin{tikzcd}
          A\times(B\times C) \arrow{r}{\pi_2} & B\times C \arrow{r}{\pi_1} & B
            \end{tikzcd}    \\
    $g := \pi_2\circ\pi_2$
        &&   \begin{tikzcd}
          A\times(B\times C) \arrow{r}{\pi_2} & B\times C \arrow{r}{\pi_2} & C
            \end{tikzcd}    
    \end{tabular}
  So then $\phi=[[p,q],g]\colon A\times(B\times C) \to (A\times B)\times C$

  We do likewise in the opposite direction: to get
  \[ \phi\inv \colon (A\times B)\times C \to A\times(B\times C) \]
  we put $\phi\inv=[h,[r,s]]$ where

  \begin{tabular}{@{} lll @{}}
    $h := \pi_1\circ\pi_1$
        &&   \begin{tikzcd}
          (A\times B)\times C \arrow{r}{\pi_1} & A\times B \arrow{r}{\pi_1} & A
            \end{tikzcd}    \\
    $r := \pi_2\circ\pi_1$
        &&   \begin{tikzcd}
          (A\times B)\times C \arrow{r}{\pi_1} & A\times B \arrow{r}{\pi_2} & B
            \end{tikzcd} \\
    $s := \pi_2$
        &\hspace{2cm}\;&   \begin{tikzcd}
          (A\times B)\times C \arrow{r}{\pi_2} & C
            \end{tikzcd}       
    \end{tabular}
\end{Answer}

This suggests that we can indeed derive all non-empty products from binary products. The only things which is missing is the \emph{empty product}, which corresponds to the terminal object. Hence we say that a category has finite products if it has a terminal object and binary products. Note that this also means that we chose a particular terminal object $1$ and for any pair of object $A$ and $B$ another object $A\times B$.

\subsection{Coproducts}
\label{sec:coproducts}

It is easy to say what are coproducts: they are simply products in the opposite category. However, it is useful to develop some intuition by looking at coproducts
\footnote{They are also called \emph{disjoint union} due to the way they are defined in set theory. However, since unions don't make sense in type theory this isn't a very good name. A better alternative to coproducts is \emph{sums}.}
 in $\cat{\Set}$ first: Given $A,B : \Set$ we define $A + B : \Set$ whose elements are $\inj_1\,a : A+B$ for $a:A$ and $\inj_2\,b : A+B$ for $b:B$ and these are the only elements. Now given $f : A \to C$ and $g : B \to C$ we define $\case{f}{g} : A + B \to C$ which allows us to do case analysis:
\begin{align*}
\case{f}{g}\,(\inj_1\,a) & = f\,a \\
\case{f}{g}\,(\inj_2\,b) & = g\,b
\end{align*}
This is the unique function which makes the following diagram commute:
\[\begin{tikzcd}[row sep=large]
& C \\
A \arrow[r,"\inj_1" below] \arrow[ur,"f"] & A + B \arrow[u,dashrightarrow,"\case{f}{g}" right] & B \arrow[l,"\inj_2"] \arrow[ul,"g" above]
\end{tikzcd}\]
Moreover this assignment is natural, i.e. $h \circ \case{f}{g} = \case{h\circ f}{h\circ g}$.
\begin{Exercise}
  Verify uniqueness and naturality of $\case{f}{g}$.
\end{Exercise}

This leads directly to the definition of binary coproducts in a category $\cat{C}$, a coproduct $A+B$ of $A$ and $B$ is given by injections:
$\inj_1 : \homC{C}{A}{A+B}$ for and $\inj_2\,b : \homC{C}{B}{A+B}$ and for any pair of morphisms $f : \homC{C}{A}{C}$ and 
$g : \homC{C}{B}{C}$ a unique morphism $\case{f}{g} : \homC{C}{A+B}{C}$ that makes the following diagram commute:
\[\begin{tikzcd}[row sep=large]
& C \\
A \arrow[r,"\inj_1" below] \arrow[ur,"f"] & A + B \arrow[u,dashrightarrow,"\case{f}{g}" right] & B \arrow[l,"\inj_2"] \arrow[ul,"g" above]
\end{tikzcd}\]
Moreover this assignment is natural, i.e. $h \circ \case{f}{g} = \case{h\circ f}{h\circ g}$.

\begin{question}
  Has the category $\cat{\omega}$, which corresponds to $\leq$ on natural numbers, got coproducts? In general what are coproducts for a preorder?
\end{question}

As for produts we say that a category has all finite coproducts if it has an initial object and all binary coproducts.

In $\Set$ and actually in many categories products and coproducts are very different, but this is not always the case. We have already seen that in $\cat{Mon}$ ths initial object and the terminal object coincide. For the following consider the category of commutative monoids $\cat{CMon}$ which is like monoids but with the additional property that $x * y = y * x$ for any object $(A,e,\_*\_)$. Given two commutative monoids  $(A,e,\_*\_),  (B,z,\_+\_)$ we define $A \oplus B$ as $(A\times B,(e,z),\_ \bullet \_)$ where $(a,b) \bullet (a',b') = (a * a',b + b')$. It is easy to see that this is a commutative monoid again.
\begin{Exercise}
Show that $A \oplus B$ is both a product and a coproduct of $A$ and $B$.
\end{Exercise}
\begin{Answer}
  \Question
    To prove that this is a product, we need to do several things.
    \begin{itemize}
      \item \textbf{Construct the projections and prove they're morphisms}

        To define a monoid morphism, we must (a) supply a function on the underlying sets; (b) prove that this function sends the netural element of the domain to the neutral element of the codomain; and (c) prove that the function respects the operations. The functions we need are of the form $A\times B\to A$ and $A\times B\to B$, so we'll just use the same projections from products in $\cat{Set}$: $\pi_1(a,b)=a$ and $\pi_2(a,b)=b$.

        These satisfy the two conditions for monoid morphisms: $\pi_1$ and $\pi_2$ send the netural element $(e,z)$ to the netural elements $e$ and $z$, respectively. They also preserve the monoid operations: if I pick any $(a,b)$ and $(a',b')$, then
        \[ \pi_1((a,b)\bullet(a',b')) = \pi_1(a * a',b + b') = a * a' = \pi_1(a,b) * \pi_1(a',b') \]
        and likewise $\pi_2((a,b)\bullet(a',b'))=\pi_2(a,b) + \pi_2(a',b')$. 
      \item \textbf{Define $[\_,\_]$ as an operation on morphisms}

      Given some other commutative monoid $(X,d,\_\diamond\_)$ and some monoid morphisms
      \[ \begin{tikzcd}  A & X \arrow[swap]{l}{f} \arrow{r}{g} & B \end{tikzcd}  \]
      we need to construct a function $X\to A\times B$ which is a monoid morphism from $(X,d,\_\diamond\_)$ to $(A,e,\_*\_)\oplus(B,z,\_+\_)$. Again, we copy from the products in $\cat{Set}$:
      \[  [f,g]\;x = (f\,x, g\,x). \]
      This is a monoid morphism: since $f$ and $g$ are monoid morphisms, they must send the netural element $d$ to the netural elements $e$ and $z$, respectively. So $[f,g]\,d=(e,z)$, the netural element of $A\oplus B$. Likewise for preservation of monoid operations:
      \begin{align*}
        [f,g]\;(x\diamond x')
          &= (f(x\diamond x'), g(x\diamond x')) \tag{Defn. $[f,g]$}\\
          &= (f(x) * f(x'),g(x\diamond x')) \tag{$f$ is a morphism}\\
          &= (f(x) * f(x'),g(x) + g(x')) \tag{$g$ is a morphism}\\
          &= (f(x),g(x)) \bullet (f(x'),g(x')) \tag{Defn. $\bullet$}\\
          &= [f,g](x) \bullet [f,g](x') \tag{Defn. $[f,g]$}
      \end{align*}
      \item \textbf{Prove that $[f,g]$ makes the triangles commute}
      Now we need that $[f,g]$ makes these two triangles commute:
      \[ \begin{tikzcd}[sep=large]
          & X \arrow[swap,bend right=7]{dl}{f} \arrow[bend left=7]{dr}{g} \arrow{d}{[f,g]}\\
          A & A\times B \arrow{l}{\pi_1}\arrow[swap]{r}{\pi_2} & B
      \end{tikzcd} \]
      To prove $\pi_1\circ [f,g]=f$, by function extensionality we take arbitrary $x:X$ and then
      \[ (\pi_1\circ[f,g])\,x = \pi_1(f\,x, g\,x) = f(x) \]
      so $\pi_1\circ [f,g]=f$. Likewise for $\pi_2$.
      \item \textbf{Prove the uniqueness of $[f,g]$}
      Finally, we must prove that $[f,g]$ is the \emph{only} morphism $k\colon X\to A\times B$ such that $\pi_1\circ k = f$ and $\pi_2\circ k=g$. Note that there could be many morphisms $X\to A\times B$: we want to prove that this is the only one \emph{making these two triangles commute}.
      
      Suppose $k\colon X\to A\times B$ is such that $\pi_1\circ k = f$ and $\pi_2\circ k=g$. To show $k$ must be $[f,g]$, we pick arbitrary $x:X$ and show $k(x)=[f,g](x)$. Then,
      \begin{align*}
        k(x)
          &= (\pi_1(k\,x),\pi_2(k\,x))\\
          &= (f(x),\pi_2(k\,x)) \tag{$\pi_1\circ k=f$}\\
          &= (f(x),g(x)) \tag{$\pi_2\circ k=g$}\\
          &= [f,g](x).
      \end{align*}
    \end{itemize}
\end{Answer}
\begin{question}
  What happens if we give up commutativity, i.e. if we consider $\cat{Mon}$ instead.
\end{question}

Products and coproducts can also be defined as an adjunction: Given a category $\cat{C}$ we write $\cat{C^2} = \cat{C}\times\cat{C}$
\footnote{Yes, this is a product in the category of categories which is not a category in our sense. We already develop a taste for higher categories\dots}
whose objects are pairs of objects and whose morphsims are pairs of morphisms in $\cat{C}$. There is a functor $\Delta : \cat{C} \to \cat{C^2}$ (called the diagonal functor) which just copies, that is $\Delta\,A = (A,A)$ and the same on morphisms. Now if $\cat{C}$ has binary products and coproducts resepctively we also have $\_\times\_, \_+\_ : \cat{C^2} \to \cat{C}$
\begin{Exercise}
  Define the morphism parts of these functors using only the properties of products / coproducts.
\end{Exercise}
Moreover, they are left and right adjoint to $\Delta$:
\[\begin{tikzcd}[row sep=large]
\cat{C} \arrow[d,"\Delta" near start] \\ 
\cat{C^2} \arrow[u,bend left = 60,"\_+\_" left,"\dashv" right=5 ] \arrow[u,bend right = 60,"\_\times\_" right,"\dashv" left = 5]
\end{tikzcd}\]
\begin{Exercise}
  Verify these adjunctions, i.e. show that
  \begin{align*}
    \cat{C}(X,Z) \times \cat{C}(Y,Z) \cong \cat{C}(X + Y,Z) \\
    \cat{C}(Z, X\times Y) \cong \cat{C}(Z,X) \times \cat{C}(Z,Y)
  \end{align*}
  are natural isomorphisms. 
\end{Exercise}

\subsection{Exponentials}
\label{sec:exponentials}

Next we look at the set of functions which is called an \emph{exponential} in category theory. One of the basic properties of functions in functional programming is currying, which is the following isomorphism:
\[\begin{tikzcd}[column sep = large]
A\times B \to C \arrow[r,"\phi_{A,B,C}",bend right,"\cong" above=11] & A\to(B \to C) \arrow[l,"\psi_{A,B,C}",bend right]
\end{tikzcd}\]
where
\begin{align*}
  &\phi_{A,B,C} : (A\times B \to C) \to (A\to(B \to C))\\
  &\phi\,\,f = \lambda x.\lambda y.f\,(x,y)\\
  &\psi_{A,B,C} : (A\to(B \to C)) \to (A\times B \to C) \\
  &\psi\,g = \lambda p.g\,(\pi_1\,p)\,(\pi_2\,p)
\end{align*}
\begin{Exercise}
  Verify that $\phi$ and $\psi$ are inverse to each other.
\end{Exercise}
If we fix $B$ this looks like an adjunction, namely the functors $\_\times B, B \to \_ : \Set \to \Set$ form an adjunction. All that is left is to check naturality:
\begin{Exercise}
  Derive the morphism part of $\_\times B$ and $B \to \_$ and
  check that $\phi_{A,B,C}$ is natural in $A,C$, that is given $f : A \to A'$ and $g : C \to C'$
\[\begin{tikzcd}[row sep = large]
   A\times B \to C \arrow[r,"\phi_{A,B,C}"] \arrow[d,"\lambda h. g\circ h \circ (f \times B)"] & A\to(B \to C) \arrow[d,"\lambda k. (B \to g)\circ k \circ f"] \\
   A'\times B \to C'  \arrow[r,"\phi_{A',B,C'}" below]  & A'\to(B \to C') 
\end{tikzcd}\]  
\end{Exercise}

This directly leads to the definition of \emph{exponentials (aka function objects)} for a category $\cat{C}$ with products, we say that for a fixed object $B:\obj{\cat{C}}$ an exponential $\expC{}{B}{\_}$ or $(\_)^B$ is the right adjoint to $\_ \times B$. We can compare $\expC{}{\_}{\_}$ with homsets $\homC{C}{\_}{\_}$ both take two objects as input but the first one returns an object while the latter returns a set. This is why exponentials are also called internal homs. Only on $\Set$ the two are actually the same.

By definition $\expC{\cat{C}}{B}{\_}$ is a covariant functor $\cat{C} \to \cat{C}$ but exponentials also give rise to a contravariant functor:
\begin{Exercise}
Show that $\expC{\cat{C}}{\_}{C}$ is a contravariant functor, i.e. 
\[\expC{}{\_}{C} : \cat{C}^\op \to \cat{C}.\]     
\end{Exercise}
%Actually $\expX{\_}{\_} : \cat{C}^\op \times \cat{C} \to \cat{C}$ is a bifunctor.

A category which has all finite products and exponentials is called a \emph{a cartesian closed category} (CCC) if it is also has coproducts it is a \emph{Bicartesian closed category (BiCCC)}. As we have already observed $\Set$ is cartesian closed and has finite coproducts hence it is a bicartesian closed category. 

The following example of a bicartesian closed category is a good justification for the notation: the category of finite sets $\cat{\Fin}$: its objects are natural numbers, and its homsets are defined as $\cat{\Fin}(m,n) = \Fin\, m \to \Fin\,n$ where $\Fin\,k = \{ i : \Nat \mid i < k\}$, i.e. $\Fin\,k$ has exactly $k$ elements. $\cat{\Fin}$ has (given $m,n : \obj{\cat{\Fin}}=\Nat$):

\begin{tabular}{|l|l|}
\hline
an initial object & $0$ \\ \hline
coproducts & $m + n$ \\ \hline
a terminal object & $1$ \\\hline
products & $m\times n$ \\\hline
exponentials & $m \to n = n^m$\\\hline
\end{tabular}

% \begin{description}
% \item[an initial object] $0$,
% \item[coproducts] $m + n$ 
% \item[a terminal object] $1$
% \item[products] $m\times n$
% \item[exponentials] $m^n$.
% \end{description}

\begin{Exercise}
Verify that these definitions indeed have the required properties.
\end{Exercise}

Another standard example for a bicartesian closed category is the category of propositions $\cat{\Prop}$: its objects are propositions and its homsets are given by implication (which correspond to functions): $\cat{\Prop}(P,Q) = P \to Q$. This category is a preorder (because all its homs are propositions) and it has (given $P,Q : \Prop$) :

\begin{tabular}{|l|l|}
\hline
an initial object & $\False$ \\\hline
coproducts & $P \vee Q$ \\ \hline
a terminal object & $\True$ \\\hline
products & $P \wedge Q$ \\\hline
exponentials & $P \to Q$\\\hline
\end{tabular}

\begin{Exercise}
Verify that these definitions indeed have the required properties.
\end{Exercise}
A bicartesian closed category which is a preorder is called a \emph{Heyting algebra}, it is a model for intuitionistic propositional logic.

% In bicartesian closed categories we can show distributivity, that is 
% \[ \phi : A \times (B + C) \cong A\times B + A \times C\]
% we could construct this isomorphism directly but there is an elegant proof using the Yoneda lemma which I will sketch. If we want to show that two objects in a category are isomorphic $A \cong B$ it is sufficient to show that there is a natural isomorphism bteween the hom functors $\cat{C}(\_,A) \cong \cat{C}(\_,B)$. This  can be shown using the Yoneda lemma for the identity functor $I :\cat{C} \to \cat{C}$:
% \begin{align*}
%   A 
%   & = I\,A\\
%   & = \int{X:\cat{C}} \cat{C}(X,A) \to  \\
%   & 
% \end{align*}

In every bicartesian closed category products and coproducts distribute as in ordinary algebra. That means we have the following isomorphisms
\begin{align*}
  0 \times A & \cong 0 \\
  (B + C)\times A & \cong B\times A + C \times A
\end{align*}
We don't need the symmetric cases since we alreay know that commutativity of products ($A \times B \cong B \times A$) and coproducts ($A + B \cong B + A$) are isomorphisms.

The easiest way to show this is to use the corollary from the Yoneda lemma: to show an isomorphism it is enough to show that the homsets out of (or into) these objects are naturally isomorphic. Let's do the 2nd isomorphism: 
\begin{align*}
  \homC{C}{(B + C)\times A}{X} 
   & \cong \homC{C}{B+C}{\expC{C}{A}{X}} \\
  & \cong \homC{C}{B}{\expC{C}{A}{X}} \times \homC{C}{C}{\expC{C}{A}{X}} \\
   & \cong \homC{C}{B\times A}{X} \times \homC{C}{C\times A}{X} \\
   & \cong \homC{C}{B \times A + C \times A}{X}
\end{align*}
and hence $(B + C)\times A \cong B\times A + C \times A$.
In each line we only used the adjunctions defining $\times$, $+$ and $\expC{}{}{}$.
\begin{Exercise}
  Prove the first isomorphism $0 \times A \cong 0$ using the same idea. Note that $\homC{C}{0}{A} \cong 1$
\end{Exercise}

Not every category with products and coproducts is distributive. A counterexample is $\cat{Mon}$ where the terminal object and the initial object agree, i.e. $0 \cong 1$. hence we can just use the first isomorphism to show that every object is isomorphic to the initial object (and the terminal):
\begin{align*}
A & \cong 1 \times A \\
& \cong 0 \times A \\
& \cong 0  
\end{align*}
But this is certainly not true in $\cat{Mon}$, hence we can conclude that $\cat{Mon}$ is not cartesian closed. 

\begin{Exercise}
  Is $\cat{\omega}$ (bi)cartesian closed? What about $\cat{\omega}^\op$?
\end{Exercise}

An important example for a bicartesian closed category is the category of presheaves of a given category $\PSh\,\cat{C}$. Its objects are functors $F : \cat{C}^\op \to \Set$ and its morphisms are natural transformations. Products (and coproducts) are calculated \emph{pointwise}: That is given $F,G : \cat{C}^\op \to \Set$
\begin{align*}
  (F \times G)\,A = F\,A \times G\,A \\
  (F + G)\, A = F\,A + G\,A
\end{align*}
\begin{Exercise}
  Verify that these are indeed products and coproducts in $\PSh\,\cat{C}$.
\end{Exercise}

But what about exponentials? We cannot have $(\expC{\PSh\,\cat{C}}{F}{G})\,A = F\,A \to G\,A$ because this wouldn't be a contravariant functor since $F$ appears on the left hand side of the exponential. The construction of the exponential is actually a nice application of the Yoneda lemma: 
If we assume that the exponential exists we have then 
\begin{align*}
(\expC{\PSh\,\cat{C}}{F}{G})\,A 
& = \int_{X:\cat{C}}C(X,A) \to (\expC{\PSh\,\cat{C}}{F}{G})\,X \\
& = \PSh\,\cat{C}(C(\_,A),\expC{\PSh\,\cat{C}}{F}{G}) \\
& = \PSh\,\cat{C}(C(\_,A) \times F,G) \\
& = \int_{X:\cat{C}}C(X,A)\times F\,X \to G\,X\\
& = \int_{X:\cat{C}}C(X,A)\to F\,X \to G\,X
\end{align*}
So the last line is a candidate for the exponential but in this case we do need to think about size. If the type of objects is not small (for example it could be $\Set$) then the set of natural transformations isn't small either. However, if we assume that $\obj{\cat{C}}$ is small we can just define
\begin{align*}
(\expC{\PSh\,\cat{C}}{F}{G})\,A & = \int_{X:\cat{C}}C(X,A)\to F\,X \to G\,X
\end{align*}
% We also need to check that this definition actually satisfies the universal property:
% \begin{Exercise}
%   Show that the above definition of $\expC{\PSh\,\cat{C}}{F}{G}$ satisfies the universal property of exponentials, i.e. for $H : \PSh\,\cat{C}$
%   \[ \PSh\,\cat{C}(H \times F,G) \simeq \PSh\,\cat{C}(H, \ex But what about exponentials? We cannot have $(\expC{\PSh\,\cat{C}}{F}{G})\,A = F\,A \to G\,A$ because this wouldn't be a contravariant functor since $F$ appears on the left hand side of the exponential. The construction of the exponential is actually a nice application of the Yoneda lemma: 
% If we assume that the exponential exists we have then 
% \begin{align*}
% (\expC{\PSh\,\cat{C}}{F}{G})\,A 
% & = \int_{X:\cat{C}}C(X,A) \to (\expC{\PSh\,\cat{C}}{F}{G})\,X \\
% & = \PSh\,\cat{C}(C(\_,A),\expC{\PSh\,\cat{C}}{F}{G}) \\
% & = \PSh\,\cat{C}(C(\_,A) \times F,G) \\
% & = \int_{X:\cat{C}}C(X,A)\times F\,X \to G\,X\\
% & = \int_{X:\cat{C}}C(X,A)\to F\,X \to G\,X
% \end{align*}
% So the last line is a candidate for the exponential but in this case we do need to think about size. If the type of objects is not small (for example it could be $\Set$) then the set of natural transformations isn't small either. However, if we assume that $\obj{\cat{C}}$ is small we can just define
% \begin{align*}
% (\expC{\PSh\,\cat{C}}{F}{G})\,A & = \int_{X:\cat{C}}C(X,A)\to F\,X \to G\,X)
% \end{align*}
We also need to check that this definition actually satisfies the universal property:
\begin{Exercise}
  Show that the above definition of $\expC{\PSh\,\cat{C}}{F}{G}$ satisfies the universal property of exponentials, i.e. for $H : \PSh\,\cat{C}$
  \[ \PSh\,\cat{C}(H \times F,G) \simeq \PSh\,\cat{C}(H, \expC{\PSh\,\cat{C}}{F}{G})\]
  i.e.
  \[ \int_{X:\cat{C}}H\,X \times F\,X \to G\,X \simeq
    \int_{X:\cat{C}}H\,X \to (\expC{\PSh\,\cat{C}}{F}{G})\,X\]
\end{Exercise}

This shows that the category or presheaves over a small category (i.e. a category with a small type of objects) is cartesian closed. However, it does for example not imply that the category of presheaves over $\Set$, i.e. functors $\cat{\Set}^\op \to \cat{\Set}$ is cartesian closed. 

Given a preorder, that is $|R|:\Set$ and $\_R\_ : |R| \to |R| \to \Prop$ that is reflexive and transitive, a propositional presheaf $P : \cat{R}^\op \to \Prop$ is just a predicate $P : |R| \to \Prop$ that is inverse monotone wrt $R$, i.e. if $P\,x$ and $y\,R\,x$ then $P\,y$. Morphisms are given by pointwise inclusion 
\[\homC{(\PSh\,R)}{P}{Q} = \forall x:|R|.P\,x \to Q\,x.\]
This is a bicartesian closed category the constructions are simplified versions of the ones for set valued presehaves, in particular 
\[(\expX{P}{Q})\,x = \forall y:|R|.y\,R\,x\to P\,y \to Q\,y \]
These categories are called \emph{Kripke models} and the fact that they are bicartesian closed corresponds to the fact that they model intuitionistic propositional logic.
\begin{Exercise}
  Use Kripke models to show that the law of the excluded middle $P \vee \neg P$ is not derivable in intuitionistic logic. 
\end{Exercise}

% Indeed they are complete
% \footnote{However, while completeness for the cartesian closed part is straightforward, the construction for coproducts relies on the decidabily of propositional logic and doesn't carry over to predicate logic. this issues can be adressed by using sheaf models instead.}
%  that is only derivable propositions are true in all Kripke models, e.g. the principle of excluded middle $P\vee\neg P$ is disproved by $|R| = \{0,1\}$ with $R\,1\,0$ and $P$ is interpreted as a predicate which only holds at $1$. Now $0$ neither validates $P$ nor $\neg P$ because this would mean that $\forall x:|R|.x\,R\,0 \to \neg P\,x$ which isn't true either. 

You may have noticed that bicartesian closed categories are a bit asymmetric. They have a right adjoint to $\_\times B$ but what about a left adjoint to $B+\_$? Let's denote such an object as $\_- B$ (minus); it should satisfy the following natural isomorphism:
\[ \homC{C}{A}{B+C} \cong \homC{C}{A - B}{C} \]
natural in $C$. 

An example for such a \emph{symmetric cartesian closed category} is the category of booleans which is similar to the category of propositions, but we use booleans instead of propositions. In particular, the type of objects is $\Bool$ and the homsets are given as 
$\homC{\Bool}{p}{q} = \T\,p \to \T\,q$ where $\T : \Bool \to \Prop$ is defined as $\T\,b = (b = \true)$. We can define products, coproducts and exponentials as for $\Prop$, i.e. by using the usual truthtable semantics. Subtraction can be defines as $p-q = p \wedge \neg q$.
\begin{Exercise}
  Check that $\cat{\Bool}$ is a symmetric cartesian closed category.
\end{Exercise}
You may notice that this example is a preorder. This is no accident because all symmetric cartesian closed categories are preorders. This is because in such a category also the dual form of distributivity hold:
\begin{align*}
  1 +  A & \cong 1 \\
  (B \times C) + A & \cong (B + A) \times (C + A)
\end{align*}
The first one is already suspicious because it implies in particular that $1+1 = 1$, which basically says that $\true = \false$ in $\Bool$. Moreover:
\begin{Exercise}
  In a bicartesian closed category with $1 = 1+1$ all morphisms are equal (i.e. it is a preorder). 
\end{Exercise}
You may wonder wether symmetry already implies excluded middle $P \vee \neg P$, which certainly holds in $\cat{\Bool}$. The answer is no, because we can extend Kripke semantics to the symmetric logic (using boolean valued presheaves over finite preorders) and uses this to show that excluded middle isn't implied. 

\begin{Exercise}
  Show that the category $\cat{1+\omega}$ is a symmetric cartesian closed category. It's definition is similar to $\cat{\omega}$ its objects are the natural numbers and a top element $\omega$ such that $i \leq \omega$ for all elements.
\end{Exercise}